\section{Site de liaison d'allèle spécifique}

\subsection{Définition}
Pour commencer un transcription, une protéïne est nécessaire. Celle-ci doit se fixer sur l'hélice. Une liaison d'allèle spécifique est ce même phénomène mais avec une protéine qui se liera d'avantages sur les allèles récessives.

\subsection{allèle préférées}

Pour trouver les places de liaisons de transcription, une unité a été crée: \textbf{PWM}: La matrice de position de poids.\\
Il s'agit d'une matrice où sont marquées les probabilités de chaque nucléotide d'apparaitre dans le site de liaison pour la transcription, ainsi que sa position.

\\
Ceci a une importance particulière car grâce à ces matrices, il a été possible de remarquer que les liaisons ce faisaient plus sur les allèles récessives que les autres.

\subsection{Recheche de site de liaison d'allèle pour la transcription pour les lymphomes}

C'est ici que la présentation recoupe avec les travaux du Dr. Wasserman.\\
\newline
En effet, grâce aux outils présentés précédemment, beaucoup de données ont été recueillis, notamment des échantillons d'ADN, d'ARN de malades atteints de lymphomes.
\\
Ensuite certaines zones du génomes ont été ciblés, et une des premières remarques est que les sites de transcriptions ont eu des taux de mutations plus élevé comparés aux séquences saines.
\\
