%%% LaTeX Template
%%% This template can be used for both articles and reports.
%%%
%%% Copyright: http://www.howtotex.com/
%%% Date: February 2011

%%% Preamble
\documentclass[paper=a4, fontsize=11pt]{scrartcl}	% Article class of KOMA-script with 11pt font and a4 format

\setcounter{secnumdepth}{4}
\setcounter{tocdepth}{4}
\makeatletter
\newcounter {subsubsubsection}[subsubsection]
\renewcommand\thesubsubsubsection{\thesubsubsection .\@alph\c@subsubsubsection}
\newcommand\subsubsubsection{\@startsection{subsubsubsection}{4}{\z@}%
                                     {-3.25ex\@plus -1ex \@minus -.2ex}%
                                     {1.5ex \@plus .2ex}%
                                     {\normalfont\normalsize\bfseries}}
\renewcommand\paragraph{\@startsection{paragraph}{5}{\z@}%
                                    {3.25ex \@plus1ex \@minus.2ex}%
                                    {-1em}%
                                    {\normalfont\normalsize\bfseries}}
\renewcommand\subparagraph{\@startsection{subparagraph}{6}{\parindent}%
                                       {3.25ex \@plus1ex \@minus .2ex}%
                                       {-1em}%
                                      {\normalfont\normalsize\bfseries}}
\newcommand*\l@subsubsubsection{\@dottedtocline{4}{10.0em}{4.1em}}
\renewcommand*\l@paragraph{\@dottedtocline{5}{10em}{5em}}
\renewcommand*\l@subparagraph{\@dottedtocline{6}{12em}{6em}}
\newcommand*{\subsubsubsectionmark}[1]{}



\usepackage[english]{babel}														% English language/hyphenation
\usepackage[protrusion=true,expansion=true]{microtype}				% Better typography
\usepackage{amsmath,amsfonts,amsthm}										% Math packages
\usepackage[pdftex]{graphicx}														% Enable pdflatex
%\usepackage{color,transparent}													% If you use color and/or transparency
\usepackage[hang, small,labelfont=bf,up,textfont=it,up]{caption}	% Custom captions under/above floats
\usepackage{epstopdf}																	% Converts .eps to .pdf
\usepackage{subfig}																		% Subfigures
\usepackage{booktabs}																	% Nicer tables
\usepackage{wrapfig} 

%%% Advanced verbatim environment
\usepackage{verbatim}
\usepackage{fancyvrb}
\usepackage[utf8]{inputenc}
\DefineShortVerb{\|}								% delimiter to display inline verbatim text


%%% Custom sectioning (sectsty package)
\usepackage{sectsty}								% Custom sectioning (see below)
\allsectionsfont{%									% Change font of al section commands
	\usefont{OT1}{bch}{b}{n}%					% bch-b-n: CharterBT-Bold font
%	\hspace{15pt}%									% Uncomment for indentation
	}

\sectionfont{%										% Change font of \section command
	\usefont{OT1}{bch}{b}{n}%					% bch-b-n: CharterBT-Bold font
	\sectionrule{0pt}{0pt}{-5pt}{0.8pt}%	% Horizontal rule below section
	}


%%% Custom headers/footers (fancyhdr package)
\usepackage{fancyhdr}
\pagestyle{fancyplain}
\fancyhead{}														% No page header
\fancyfoot[C]{\thepage}										% Pagenumbering at center of footer
\fancyfoot[R]{\small \texttt{Master II 2014-2015}}	% You can remove/edit this line 
\renewcommand{\headrulewidth}{0pt}				% Remove header underlines
\renewcommand{\footrulewidth}{0pt}				% Remove footer underlines
\setlength{\headheight}{13.6pt}



%%% Equation and float numbering
\numberwithin{equation}{section}															% Equationnumbering: section.eq#
\numberwithin{figure}{section}																% Figurenumbering: section.fig#
\numberwithin{table}{section}																% Tablenumbering: section.tab#


%%% Title	
\title{ \vspace{-1in} 	\usefont{OT1}{bch}{b}{n}
		\huge \strut Discovery of metabolic gene mutations causing intellectual delay \strut \\
		\Large \bfseries \strut Wyeth W. Wasserman \strut
}
\author{ 									
	\usefont{OT1}{bch}{m}{n} David Aubert\\		
	\usefont{OT1}{bch}{m}{n} Ahmed Rafik\\		
	\usefont{OT1}{bch}{m}{n}University of Montpellier\\	
	\usefont{OT1}{bch}{m}{n}Bio-informatique\\
}
\date{05 jan 2015}


%%% Begin document
\begin{document}
\maketitle


\section{Etude du génome}

\subsection{Pourquoi cette étude}

Un laboratoire a réussi à mettre en évidence le lien entre des maladies entrainant un retard intellectuelle et des mutation génétique, ce qui a poussé le docteur Wasserman et son équipe à se pencher sur le sujet.
Il commence par décrire les différents symptomes que l'on peut observer chez une fratrie de nouveau nées comme par exemple, à la naissance, des difficultés respiratoires qui réapparaissent à 2 ans et demi ainsi qu'à 3 ans et demi.
Rapidement, ils ont pu écarté de nombreuses thèses et se concentrer sur les pistes qui nous intéressent.

\subsection{WGS VS WES}

\textbf{Défintitions simplifiées}
The Whole genome sequencing : c'est un procédé qui permet de déterminer l'ensemble des séquences d'ADN du génome d'un organisme donné.
The Whole exome sequencing : c'est un procédé qui va selectionner l'ensemble des séquences d'ADN qui encode des protéines, et ensuite va séquencer celles-ci.

Le procédé WGS est donc plus lourd a utilisé que WES ( WGS = ~WESx6 bp)

\subsection{le gène CA5A}
D'abord, nous devons définir ce qu'est l'anhydrase carbonique en général qui l'enzyme secrété par CA5A: 
C'est une enzyme présente à la surface plasmique intracellulaire des globules rouges qui permet transforme le gaz carbonique $CO_2$ en $H_2CO_3$.
le CA5A est donc un gène de la famille des Anhydrases Carboniques. Il encode l'anhydrase carbonique dans les cellules mitochondrial.

Malgré le rôle de CA-VA dans le métabolisme intermédiaire extérieur des crises mortelles, on peut observer des phénotypes légers (léger retard dans le developement mental ou dans la croissance par exemple)
c'est peut s'expliquer éventuellement de la manière suivante :
Il est possible qu'il y ait un chevauchement fonctionnel de la production de bicarbonate avec l'enzyme mitochondriale CA5B. CA-VA étant une enzyme néonatale, elle devient moins important avec l'âge.

\subsection{L'approche de l'équipe}

L'équipe a eu une approche très classique pour aborder ce probleme qui leur donna les résultat suivants :
un premier séquensage avec WES sur 67 familles révéla que 3 n'était pas génétique tandis que 64 l'était. Parmis ceux-ci, ils ont pu identifier des mutations chez 52 familles et pour les 12 restantes, il fallu utiliser WGS par manque de résultat.

\section{Medical Subject Heading Over–representation Profiles}
\subsection{Définition}
MeSHOP est une application très complète.Celle-ci permet de centraliser les recherches et d'indexer les publications scientifiques. Celli ci englobe des thématiques allant de la recherche génétique à la chimie en passant par les maladies, ce qui interesse l'auteur.

\subsection{fonctionnement}
Le moteur de cette application repose sur une base de données MEDLINE et le moteur de recherche PubMed.

\iffalse
\includegraphics[width=40mm, height=15mm, scale=0.5]{medline}
\includegraphics[width=25mm,scale=0.5]{pubmed}
\fi

\subsubsection{Annotations}
Chaque article devra être annoté avec un vocabulaire particulier. En effet, c'est cette annotation qui permettra une recherche efficace. Le système MeSH a une liste de vocabulaire (\textit{PMIDs}) qui servira d'index pour le fichier.

\subsection{Editer un article}
Après l'écriture d'un article, il est nécessaire de le publier. Et c'est à ce moment là que va servir le système d'annotation. En effet une lecture par un robot de l'article sera fait et permettra, en fonction du vocabulaire utilisé, de classer l'article dans des catégories et de l'indexer pour des recherches.
\\
\begin{figure}{l}{}
\centering
\includegraphics{nuagepoint}
\caption{exemple d'un nuage de mot}
\end{figure}


Chaque mot sera classé en fonction de son nombre d'ocurence dans les \textit{PMIDs}


\subsection{Recherche}
Un point important est la thématique recherchée. En effet, comme vu précédemment, les sujets d'articles englobent un grand nombre de thématique (génétique, maladies, chimie etc...) et donc le contexte peut changer pour des mêmes mots ou combinaisons de mots.
\input{transcriptionregulation}
\section{Site de liaison d'allèle spécifique}

\subsection{Définition}
Pour commencer un transcription, une protéïne est nécessaire. Celle-ci doit se fixer sur l'hélice. Une liaison d'allèle spécifique est ce même phénomène mais avec une protéine qui se liera d'avantages sur les allèles récessives.

\subsection{allèle préférées}

Pour trouver les places de liaisons de transcription, une unité a été crée: \textbf{PWM}: La matrice de position de poids.\\
Il s'agit d'une matrice où sont marquées les probabilités de chaque nucléotide d'apparaitre dans le site de liaison pour la transcription, ainsi que sa position.

\\
Ceci a une importance particulière car grâce à ces matrices, il a été possible de remarquer que les liaisons ce faisaient plus sur les allèles récessives que les autres.

\subsection{Recheche de site de liaison d'allèle pour la transcription pour les lymphomes}

C'est ici que la présentation recoupe avec les travaux du Dr. Wasserman.\\
\newline
En effet, grâce aux outils présentés précédemment, beaucoup de données ont été recueillis, notamment des échantillons d'ADN, d'ARN de malades atteints de lymphomes.
\\
Ensuite certaines zones du génomes ont été ciblés, et une des premières remarques est que les sites de transcriptions ont eu des taux de mutations plus élevé comparés aux séquences saines.
\\

\section{Finalité}


\begin{hspace}{1cm}
Comme
\end{hspace}
expliqué, les sites de transcriptions sont importants dans l'expression du génome. Des malades attendent des avancées médicales mais la technologie bloque sur le séquencage ADN.  La recherche est mobilisée. Malgré l'aspect financier, il existe des initiatives qui permettent de partager des résultats, des expériences et d'aider la recherche (MeSH). 
\newline

\begin{hspace}{1cm}
Mais
\end{hspace} malgré cela la recherche bloque sur le séquencage du génome, pour se subsituer à cet obstacle de nouvelles idées émergent et au lieu de faire une recherche brute, des méthodes plus affinées sont implémentés par le Dr. Wasserman et dans le cas où le problème ne serait pas entièrement génétique mais dans son expression, via les sites de transcriptions. 
\newline

\subsection{perspectives}





\iffalse 
\subsubsection{Heading on level 3 (subsubsection)}
Nulla consequat massa quis enim. Donec pede justo, fringilla vel, aliquet nec, vulputate eget, arcu. In enim justo, rhoncus ut, imperdiet a, venenatis vitae, justo. Nullam dictum felis eu pede mollis pretium. Integer tincidunt. Cras dapibus. Vivamus elementum semper nisi. Aenean vulputate eleifend tellus. Aenean leo ligula, porttitor eu, consequat vitae, eleifend ac, enim.

\paragraph{Heading on level 4 (paragraph)}
Lorem ipsum dolor sit amet, consectetuer adipiscing elit. Aenean commodo ligula eget dolor. Aenean massa. Cum sociis natoque penatibus et magnis dis parturient montes, nascetur ridiculus mus. Donec quam felis, ultricies nec, pellentesque eu, pretium quis, sem. Nulla consequat massa quis enim. 


\section{Lists}
\subsection{Example for list (itemize)}
\begin{itemize}
	\item First item in a list 
	\item Second item in a list 
	\item Third item in a list
\end{itemize}

\subsubsection{Example for list (3*itemize)}
\begin{itemize}
	\item First item in a list 
		\begin{itemize}
		\item First item in a list 
			\begin{itemize}
			\item First item in a list 
			\item Second item in a list 
			\end{itemize}
		\item Second item in a list 
		\end{itemize}
	\item Second item in a list 
\end{itemize}

\subsection{Example for list (enumerate)}
\begin{enumerate}
	\item First item in a list 
	\item Second item in a list 
	\item Third item in a list
\end{enumerate}

\subsubsection{Example for list (3*enumerate)}
\begin{enumerate}
	\item First item in a list 
		\begin{enumerate}
		\item First item in a list 
			\begin{enumerate}
			\item First item in a list 
			\item Second item in a list 
			\end{enumerate}
		\item Second item in a list 
		\end{enumerate}
	\item Second item in a list 
\end{enumerate}

\section{Mathematics}
Let's display some math:
\begin{align} 
	\begin{split}
	(x+y)^3 	&= (x+y)^2(x+y)\\
					&=(x^2+2xy+y^2)(x+y)\\
					&=(x^3+2x^2y+xy^2) + (x^2y+2xy^2+y^3)\\
					&=x^3+3x^2y+3xy^2+y^3
	\end{split}					
\end{align}

\begin{align}
	A = 
	\begin{bmatrix}
	A_{11} & A_{21} \\
  	A_{21} & A_{22}
	\end{bmatrix}
\end{align}
\fi

\end{document}