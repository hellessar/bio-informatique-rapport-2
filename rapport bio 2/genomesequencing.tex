\section{Etude du génome}

\subsection{Pourquoi cette étude}

Un laboratoire a réussi à mettre en évidence le lien entre des maladies entrainant un retard intellectuelle et des mutation génétique, ce qui a poussé le docteur Wasserman et son équipe à se pencher sur le sujet.
Il commence par décrire les différents symptomes que l'on peut observer chez une fratrie de nouveau nées comme par exemple, à la naissance, des difficultés respiratoires qui réapparaissent à 2 ans et demi ainsi qu'à 3 ans et demi.
Rapidement, ils ont pu écarté de nombreuses thèses et se concentrer sur les pistes qui nous intéressent.

\subsection{WGS VS WES}

\textbf{Défintitions simplifiées}
The Whole genome sequencing : c'est un procédé qui permet de déterminer l'ensemble des séquences d'ADN du génome d'un organisme donné.
The Whole exome sequencing : c'est un procédé qui va selectionner l'ensemble des séquences d'ADN qui encode des protéines, et ensuite va séquencer celles-ci.

Le procédé WGS est donc plus lourd a utilisé que WES ( WGS = ~WESx6 bp)

\subsection{le gène CA5A}
D'abord, nous devons définir ce qu'est l'anhydrase carbonique en général qui l'enzyme secrété par CA5A: 
C'est une enzyme présente à la surface plasmique intracellulaire des globules rouges qui permet transforme le gaz carbonique $CO_2$ en $H_2CO_3$.
le CA5A est donc un gène de la famille des Anhydrases Carboniques. Il encode l'anhydrase carbonique dans les cellules mitochondrial.

Malgré le rôle de CA-VA dans le métabolisme intermédiaire extérieur des crises mortelles, on peut observer des phénotypes légers (léger retard dans le developement mental ou dans la croissance par exemple)
c'est peut s'expliquer éventuellement de la manière suivante :
Il est possible qu'il y ait un chevauchement fonctionnel de la production de bicarbonate avec l'enzyme mitochondriale CA5B. CA-VA étant une enzyme néonatale, elle devient moins important avec l'âge.

\subsection{L'approche de l'équipe}

L'équipe a eu une approche très classique pour aborder ce probleme qui leur donna les résultat suivants :
un premier séquensage avec WES sur 67 familles révéla que 3 n'était pas génétique tandis que 64 l'était. Parmis ceux-ci, ils ont pu identifier des mutations chez 52 familles et pour les 12 restantes, il fallu utiliser WGS par manque de résultat.
